% \section{Introduction}

% \len{Briefly introduce the problem you seek to solve (i.e., detection of age related linguistic features from dialogue and discourse), your hypotheses, and give an overview of the chapter (i.e., data, methods and models, results, and analyses).}

In this chapter we report experiments aimed at age detection from text, and the components involved. The problem we tackle in this first phase of experiments is automated detection of age-related linguistic patterns in dialogue and discourse, using current text-based NLP models. Being able to detect and investigate these linguistic differences is important for controlled dialogue generation, because it suggests that adapting automated conversational responses to a user's age is possible. Moreover, it can provide us with insights about which linguistic features are most salient for distinguishing between, and adapting to, different age groups. We expect that the classification models are able to reliably detect age-related differences in both transcribed dialogue and discourse, and the most informative differences to lie at the syntactic-level.

The following section describes the two datasets used for these experiments. There we provide descriptive statistics, examples, and comparisons between the datasets. Section \ref{sec:exp1_methods_exp_setup} covers the problem description in more detail, along with the models used, and our experimental setup. The classification results are presented in Section \ref{sec:exp1_results}. Then for the dialogue classification models, Section \ref{sec:exp1_analyses} contains both quantitative and qualitative analyses of the results.