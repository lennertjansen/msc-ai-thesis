\section{Recap, Discussion \& Conclusion}

\textbf{Keep in mind when writing this section}
\begin{itemize}
    \item Goals of this section: a discussion/conclusion which also paves the way to and motivates Experiment 2.
    \item Recap the results and what we have learned from the analysis.
    \item (Based on the results and analyses) Make hypotheses that are relevant for the following experiment
    \item However, don't repeat (too much) what is already stated in the next section (i.e., the introduction to Experiment 2)
    \item Do I need a discussion here, given that there is a general discussion (i.e., for both experiments) at the end of the thesis?
\end{itemize}

Useful phrases from age detection paper (EMNLP submission)

\begin{itemize}
    \item We investigated whether, and to what extent, NLP models can detect age-related linguistic features in dialogue data.
    \item We showed that, in line with what we observed for discourse, state-of-the-art models are capable of doing so with a reasonable accuracy, in particular when the dialogue fragment is long enough to contain discriminative signal.
    \item At the same time, differently from discourse, we found that much simpler models based on n-grams achieve comparable performance, which suggests that, in dialogue, ‘local’ features can be indicative of the language of speakers from different age groups.
    \item We showed this to be the case, with both lexical and stylistic cues being informative to these (and possibly all) models in performing the task.
\end{itemize}

\textbf{How will these results inform the controlled generation experiments?}

\begin{itemize}
    \item Do I need this? It's literally one of the first things I talk about on the next page. (maybe have a call with sandro about it)
\end{itemize}