\begin{enumerate}
    \item To what extent can a classifier detect age-related linguistic differences in natural language? And which features are most helpful in age-group detection?
    \begin{enumerate}
        \item Do they (i.e., the linguistic or latent features exploited by the classifier) match the age-related informative features reported in previous work?
        \begin{itemize}
            \item \textbf{Hypothesis:} I expect the age-group of a text's producer to be very detectable. Also, I predict efficacy of the models to increase with their complexity. Based on \cite{pennebaker2003words}, my expectations say that complexity of word use increases with age, self-references decreases with seniority, along with the use of negative affect words. Furthermore, I also expect the importance for determining of people's linguistic characteristics to vary with the type of text. Namely, I expect the differences in word use between more formally written texts (such as blogs or articles), and usually informal discourse, to be reflected in the importance of linguistic features.
            \item \textbf{Implications:} This question aims to lay the foundation of the later research objectives by first establishing whether a discriminator can successfully categorise discourse into correct user age brackets (this is somewhat of a redundant question, because lots of literature suggests the positive). I plan to build increasingly complex classifiers, starting from a simple BoW-model, progressing to embedding-based architectures. The next entailed goal is to investigate which aspects of users' word use seem to play a role in determining the corresponding age groups. NB: these salient characteristics could also express themselves in the latent spaces of neural classifiers. Finding and understanding such features will not only help me better understand the relationship between age-group and word use, but also provide guidelines for the development of controllable conversational response generators later in the project.
        \end{itemize}
    \end{enumerate}
    
    \item Can large-scale language models (LMs), e.g. GPT-2, be leveraged for text generation, controlled for age-groups?
    \begin{enumerate}
        \item What role does the used data play in the differences in output and performance between regular GPT-2 and controllable GPT-2?
        \begin{itemize}
            \item \textbf{Hypothesis:} Based on \cite{dathathri2019plug, pennebaker2003words}, I expect they can. Assuming that age-group is distinctive enough a trait of word use for it to notably affect writing style when controlled for (see previous hypothesis), age then simply becomes a topic or writing style input that neatly fits in the the Plug-and-Play setup proposed by \citeauthor{dathathri2019plug}. 
            \item \textbf{Implications:} \textbf{TODO: EXPAND} to confirm the PPLM setup with age as an attribute input. First step in achieving adaptive dialogue generation, as "vanilla" text generation is much less restrictive.
        \end{itemize}
    \end{enumerate}
    \item To what degree is such a controlled text generation (CTG) model successful in dialogue that is adaptive w.r.t. age, such that it has a detectable effect on the perception of the user?
    \begin{itemize}
        \item \textbf{Hypothesis:} Yes, based on PPLM and DialoGPT paper and previous hypothesis
        \item \textbf{Implications:} \textbf{TODO: expand} this is in alignment with the objectives of the Human(e) AI research project. The ultimate challenge of this thesis will be to leverage the grammatical fluency of GPT-2 and the inexpensive tunability of \citeauthor{dathathri2019plug}'s PPLM setup for a novel application: adaptive neural conversational response generation.
    \end{itemize}
\end{enumerate}

\textbf{What is novel or contributing about your research?}
\begin{itemize}
    \item \textit{To the best of my knowledge, there is no work on controllable text generation for dialogue responses. Moreover, I also haven't found instances of neural conversational response systems adapting word use to the user's age.}
\end{itemize}
