\begin{enumerate}
    \item To what extent can a classifier automatically detect age-related linguistic differences in natural language? And which features are most helpful in age-group detection?
    \begin{enumerate}
        \item Do they (i.e., the linguistic or latent features exploited by the classifier) match the age-related informative features reported in previous work?
        \begin{itemize}
            \item \textbf{Hypothesis:} I expect the age-group of a text's producer to be reliably detectable. Also, I suspect efficacy of the models to increase with their complexity, with complexity going from $n$-gram, to RNN-based, to Transformer-based. I suspect linguistic trends in age found in \cite{pennebaker2003words} and \cite{schler2006effects} to persist in the feature importance of the lexically focused $n$-gram models. Moreover, I expect the encoder-decoder architectures to uncover more latent age-related patterns in language that eluded earlier automated age-detection work, e.g., \cite{nguyen-etal-2011-author}. Finally, I suspect the differences in language use between age-groups to lie more at the structural than lexical level.
            \item \textbf{Implications:} This question aims to lay the foundation of the later research objectives by first establishing whether a discriminator can successfully categorize discourse into correct user age-brackets. I plan to build increasingly complex classifiers, starting from a bag-of-words (BoW) $n$-gram model, progressing to embedding-based architectures (LSTM and BERT). The next entailed goal is to investigate which aspects of users' language use seem to play a role in determining the corresponding age groups. NB: these salient characteristics could also express themselves in the latent spaces of neural classifiers. Finding and understanding such features will not only help me better understand the relationship between age-group and word use, but also provide guidelines for the development of controllable conversational response generators later in the project.
        \end{itemize}
    \end{enumerate}
    
    \item Can large-scale language models (LMs), i.e. GPT-2 \citep{radford2019language} or DialoGPT \citep{zhang2019dialogpt}, be leveraged for text generation, controlled for age-groups?
    \begin{enumerate}
        \item What role does the used data play in the differences in output and performance between the base (conversational) language models and their controlled counterparts?
        \begin{itemize}
            \item \textbf{Hypothesis:} Based on \cite{madotto-etal-2020-plug, dathathri2019plug, pennebaker2003words}, I expect they can. Assuming that age-group is distinctive enough a trait of language use for it to notably affect writing style when controlled for (see previous hypothesis), age can be seen as an attribute input that neatly fits in the the plug-and-play setup proposed by \citeauthor{dathathri2019plug}. My research is conducted using \textbf{(1)} a dataset of blogs labeled by author age, and \textbf{(2)} a collection of transcribed conversations labeled by speaker age. I do not expect the controllability of the guided generation model to be impacted by the fact that one is conversational and the other is not. The ways that I do expect the choice of data for the controllable system to impact its performance and implications are differences in the mode of discourse (written versus spoken), recency (blogs are from 2004, conversations from 2014), and input length. Finally, it is to be expected from \cite{dathathri2019plug, madotto-etal-2020-plug} that the fluency of the base language model will tend to deteriorate after controlling for either dataset.
            \item \textbf{Implications:} To confirm the PPLM setup with age as an attribute input. First step in achieving adaptive dialogue generation, as non-conversational text generation is less restrictive.
        \end{itemize}
    \end{enumerate}
    \item To what degree is such a controlled text generation (CTG) model successful in dialogue that is adaptive w.r.t. age, such that it has a detectable effect on the perception of the user?
    \begin{itemize}
        \item \textbf{Hypothesis:} I expect the perceived quality of the dialogues with the adaptive system to be adequate, but decrease for increasing numbers of conversational turns with a user. It can also be expected that evaluating if generated responses are representative of age-specific vernacular will be more decisive with automatic than human evaluation. Nevertheless, I expect the age-differences in generated language to be noticeable by humans. 
        \item \textbf{Implications:} This is in alignment with the objectives of the Human(e) AI research project. The ultimate challenge of this thesis will be to leverage the grammatical fluency of GPT-2 or DialoGPT and the inexpensive tunability of \citeauthor{dathathri2019plug}'s PPLM and \cite{madotto-etal-2020-plug}'s PPCM setups for age-adaptive neural conversational response generation.
    \end{itemize}
\end{enumerate}

% \textbf{What is novel or contributing about your research?}
% \begin{itemize}
%     \item \textit{To the best of my knowledge, there is no work on controllable text generation for dialogue responses. Moreover, I also haven't found instances of neural conversational response systems adapting word use to the user's age.}
% \end{itemize}
